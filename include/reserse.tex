% !TEX root = ../DP_Vik_Tomas_2013.tex
% pokyny
% Z uživatelského hlediska je velmi výhodné před nákupem zboží prozkoumat trh pomocí internetových srovnávačů. Zatím ovšem není snadno dostupná služba, která by uživateli umožnila vytvořit si hypotetický seznam zboží, které zvažuje koupit.

% *Vytvořte rešerši stávajících systémů zabývajících se organizací nákupních seznamů.
% *Zjistěte možnosti a strategii napojení na systémy zabývající se srovnáním a sledováním cen.
% *V souladu s touto rešerší poté analyzujte, navrhněte a implementujte webovou aplikaci, která bude umožňovat přidávat, sledovat, prioritizovat a kategorizovat zboží, zobrazovat jeho cenu a její vývoj.
% *Webovou aplikaci implementujte v jazyce Ruby ve vhodném frameworku.
% *Pro prioritizaci navrhněte a implementujte vhodné algoritmy.
% *API webové aplikace bude navrženo s ohledem na budoucí integraci s dalšími systémy (např. srovnávače cen apod.).
\chapter{Rešerše}
Tato kapitola se zabývá popisem problematiky této práce. V této kapitole bude postupně probrána problematika práce ze tří úhlů pohledu.

První část se zabývá přímými konkurenty aplikace, která je předmětem této práce. Bude se tedy zabývat webovými aplikacemi, které umožňují organizaci nákupníhch seznamů.

Druhá část se zabývá aplikacemi, které mají nějakou významnou funkčnost vhodnou pro výslednou webovou aplikaci. Jedná se o aplikace, které se nějakou svojí částí specializují na organizaci informací.

Třetí část se zabýva technologickým aspektem práce. Budou shrnuty technologie, na kterých je v dnešní době možné postavit webové aplikace. Na konci podkapitoly budou shrnuty slabé a silné stránky jednotlivých technologií.

\section{Nákupní seznamy}
Tato kapitola popisuje funkčnost hotových aplikací zabývajících se organizací nákupního seznamu. Všechny zmíněné aplikace jsou již v produkcí a jsou zaštítěny velkými internetovými společnostmi.

\subsection{Google ShoppingList}
Webová aplikace doplňující službu Google Nákupy. Podporuje jednoduché přidání zboží. U přidaných položek zobrazuje následující informace a umožňuje podle těchto informací i řadit položky na seznamu:
\begin{itemize}
\item \textbf{Datum přidání položky}
\item \textbf{Nejnižší aktuální cena} - Služba Google Nákupy má informace o ceně získané z většího množství obchodů. U věci na seznamu zobrazuje nejnižší možnou cenu. Jiná než tato cena nelze zobrazit.
\item \textbf{Hodnocení produktu}
\end{itemize}
Zboží přidané do seznamu se dělí do dvou kategorií resp. seznamů. První seznam se nazývá Shopping List a jedná se o privátní nákupní seznam. Položky v tomto seznamu se zobrazují puze uživateli, kterému seznam patří. Druhý seznam se nazývá seznam přání a je možné na něj zkopírovat odkaz. Tento odkaz následně uživatel odešle komukoli, s kým chce svůj seznam přání sdílet. Položky lze přesouvat mezi seznamy tlačítkem \emph{Sdílet}/\emph{Zrušit sdílení}.
Zboží v seznamu přání je primárně určeno pro ostatní uživatele, aby jej mohli koupit uživateli, který seznam přání vytvořil. Například jako vánoční, narozeninový, nebo svatební seznam přání. Seznam přání je vidět na obrázku \ref{fig:google-shoppinglist}

Další funkčnost, kterou nabízí tento seznam je:
\begin{itemize}
\item Přidání poznámky k položce - uživateli je umožněno přidat poznámku k položce na seznamu.  Tato poznámka je následně zobrazena u položky v hlavním přehledu.
\item Odebrání položky ze seznamu
\item Sdílení/Zrušení sdílení - viz. předchozí odstavec
\end{itemize}

\begin{figure}[htb]
\begin{center}
\includegraphics[width=120mm]{./pictures/google-shopping-list.png}
\caption{Google ShoppingList - základní obrazovka}
\label{fig:google-shoppinglist}
\end{center}
\end{figure}

\subsection{Amazon Wish List}
Amazon Wish List je funkcionalita poskytovaná komplementárně k internetovému obchodu Amazon.com. Služba umožňuje vytvářet a sdílet seznamy přání. Pro využívání této služby musí být uživatel přihlášen. Pokud se nepřihlášený uživatel pokusí přidat nějaký předmět do seznamu přání, je přesměrován na stránku přihlášení.

\subsubsection{Druhy přání}
Tato sekce se zabývá druhy přání ve službě Amazon Wish List. Přáním se v této službě rozumí tři věci:
\begin{itemize}
\item Produkt z internetového obchodu Amazon.com
\item Internetová stránka mimo Amazon.com
\item Nápad na přání (tzv. idea)
\end{itemize}

\paragraph{Produkt internetového obchodu Amazon.com}
\label{par:produkt-amazon}
Toto přání reprezentuje jedna k jedné produkt v internetovém obchodu Amazon.com. Toto přání vznikne tak, že přihlášený uživatel klikne na stránce produktu v obchodě Amazon.com na tlačítko \emph{Add To Wish List}. U takovéhoto přání se ukazuje minimální možná cena, popis i obrázek. Tedy maximální množství informací.

\paragraph{Internetová stránka mimo Amazon.com}
Toto přání reprezentuje jakoukoliv internetovou stránku. Takovéto přáni vznikne jedině pomocí pluginu Amazon Wish List Button. O této funkcionalitě pojednává samostatná podkapitola.

\paragraph{Nápad na přání}
Nápad na přání je funkcionalita, která umožní uživateli zadat nápad na přání pouze jako text (název přání). Tento nápad se uloží mezi ostatní přání. Místo obrázku přání je ovšem zobrazen obrázek nalepovacího štítku a na něm je napsáno "I Want". Nápad na přání je vidět na obrázku \ref{fig:amazon-wishlist-idea}. Každý nápad na přání má u sebe tlačítko \emph{Top search results} které najde pro popis nápadu produkty z obchodu Amazon.com. Takto nalezené produkty je možné okamžitě přidávat do seznamu. Nápad na přání přitom v seznamu zůstává, dokud jej uživatel neodstraní.

\begin{figure}[htb]
\begin{center}
\includegraphics[width=100mm]{./pictures/amazon-wishlist-idea.png}
\caption{Nápad na přání v Amazon Wish List}
\label{fig:amazon-wishlist-idea}
\end{center}
\end{figure}

\subsubsection{Amazon Wish List Button}
Amazon Wish List poskytuje plugin\footnote{Také zásuvný modul - software, který nepracuje samostatně, ale jako doplňkový modul jiné aplikace a rozšiřuje její funkčnost.} do všech majoritních internetových prohlížečů\cite{website:amazon:plugin}. Po nainstalovani pluginu do prohlížeče si může uživatel přidat tlačítko \emph{Amazon Wish List Button} do ovládacího panelu prohlížeče. Po stisku tohoto tlačítka se otevře dialog. (Obrázek \ref{fig:amazon-wishlist-plugin})

\begin{figure}[htb]
\begin{center}
\includegraphics[width=100mm]{./pictures/amazon-wishlist-plugin.png}
\caption{Dialog zobrazeny po stisku Amazon Wish List Button}
\label{fig:amazon-wishlist-plugin}
\end{center}
\end{figure}

Přidat je možné buď produkt z obchodu Amazon.com. V tomto případě se načtou všechna data, která se u přání ukladájí, a není rozdíl v použití pluginu oproti stisknutí tlačítka "Add to Wish List" na stránce produktu (viz. kapitola \ref{par:produkt-amazon}).

Dále plugin podporuje funkci přidání přání v podobě jakékoli stránky. Tedy je možné přidat do Amazon Wish List například zboží z českého internetového obchodu. Jako obrázek k přání si uživatel může zvolit libovolý obrázek ze stránky. Jako název přání se použije titulek stránky\footnote{Obsah HTML tagu <title> ve zdrojovém kódu stránky}, popisek se k přání žádný nepřidává. Uživatel si ještě k přání muže doplnit cenu před tím než ho uloží do seznamu.

\subsection{WishList.com}
WishList.com je webová aplikace zaměřená primárně na vytváření seznamu přání. U každého seznamu podporuje sdílení a také podporuje nastavení události k seznamu a data této události. Většina seznamů má tyto hodnoty vyplněny a jsou to například seznamy přání k narozeninám, vánocům nebo seznam svatebních darů. Jako zdroj dat používá internetový srovnávač PriceGrabber.com, který je popsán v rešerši srovnávačů.

WishList.com neumožňuje přihlášení pomocí OpenID. Uživatel se může přihlásit pomocí svého Facebook účtu, nebo zaregistrovat klasickou metodou, kdy zadá svůj email, přihlašovací jméno a heslo.

Uživatel může vytvářet nové seznamy přání a nová přání. Při vytvoření nového seznamu zadává uživatel název seznamu, obrázek seznamu, omezení přístupu (veřejný, pouze přátelé, soukromý), popis seznamu, osobní poznámky, osobu, která si věci přeje, název a datum události. Pouze název seznamu je povinné pole. Seznam je navíc možné zabezpečit heslem.

Při přidávání přání je možné zadat prodejce předmětu, název předmětu, popis předmětu, cenu, množství, prioritu, poznámky k přání, seznam přání do kterého bude přání přidáno, příjemce přání a poté URL produktu a URL obrázku produktu. Povinným polem je pouze název předmětu.

Vytvořený seznam je možné sdílet pomocí URL. Pokud si seznam prohlíží jiný uživatel než ten, který jej vytvořil, může si tento uživatel zarezervovat přání, což znamená že koupí předmět a dá jej uživateli, který si jej přál.

Seznam přání i jednotlivá přání je možné sdílet s ostatními uživateli pomocí tlačitka \emph{Share this WishList} resp \emph{Share this Wish}. Sdílení je možné pomocí služeb Facebook, Myspace, Google, Twitter, Email a dalších více než 300 sociálních služeb. Na sdílení je pravděpodobně použit nějaký plugin.

Aplikace WishList podporuje vyhledávání zboží pomocí více zmíněného PriceGrabber.com. Pokud má zboží pouze jednoho prodejce, přesměruje uživatele tlačítko \emph{Buy} u přání přímo na stránku prodejce. Jinak je uživatel přesměrován na stránku s výběrem prodejců. Tato stránka se nachází stále na WishList.

Po přidání přání, které bylo nalezeno vyhledáváním je uživateli zobrazena zpráva, která se ptá jestli chce uživatel opravdu čekat a nekoupí si své přání rovnou. Tato výzva je vidět na obrázku \ref{fig:wishlist-buynow}.

\begin{figure}[htb]
\begin{center}
\includegraphics[width=100mm]{./pictures/wishlist-buynow.png}
\caption{Výzva k okamžitému zakoupení přání.}
\label{fig:wishlist-buynow}
\end{center}
\end{figure}

Přehled přání je vidět na obrázku \ref{fig:wishlist-wishlist}.

\begin{figure}[htb]
\begin{center}
\includegraphics[width=100mm]{./pictures/wishlist-wishlist.png}
\caption{Hlavní přehled přání na stránce WishList.com}
\label{fig:wishlist-wishlist}
\end{center}
\end{figure}

\subsubsection{Nedostatky WishList.com}
Cena u přání jde zadat pouze v dolarech. WishList umožňuje vyhledat zboží pro přání u kterých není uvedena URL produktu, ale vyhledávání nefunguje kompletně, nenašlo nic pro výraz "iphone", pro který PriceGrabber (zdroj dat pro WishList) nalezne několik stovek výsledků.

\section{Srovnávače cen}
Tato kapitola poskytuje rešerši produktů známých jako srovnávače cen. Cílem těchto produktů je poskytnout uživateli širší pohled na trh. Tyto produkty umožňují u jednoho druhu zboží (např. Televizor) srovnat jeho cenu v několika internetových obchodech. V rešerši jsou zpracovány jako potencionální zdroje dat pro výslednou webovou aplikaci.

\subsection{Standartní funkce všech srovnávačů}
V této sekci jsou popsány tři základní funkce, které podporují všechny zmíněné srovnávače cen. Lze říci že se jedná o funkce, které definují srovnávač cen jako takový.

\subsubsection{Vyhledávání}
Každý ze zmíněných vyhledávačů podporuje fultextové vyhledávání zboží\footnote{(z ang. full – celý, plný a text) speciální způsob vyhledávání informací v databázích nebo v textových souborech, které jsou obvykle předem připraveny, tj. indexovány, aby bylo možno nalézt libovolné slovo (řetězec znaků) v nejkratším možném čase.}. Toto vyhledávání probíhá nad databází veškerých výrobků, které aplikace (srovnávač) získala prohledáváním ostatních stránek. Vyhledávány jsou jednotlivé výrobky, u nichž jsou agregovány ceny z jednotlivých obchodů, tzn. že výrobek se zobrazí pouze jednou, přesto že srovnávač má informace o několika obchodech, které jej poskytují.

Dále umožňují srovnávače vyhledávání po kategoriích. Na domovské stránce srovnávače je uživateli nabídnut seznam kategorií a významných podkategorií. Uživatel po rozkliknutí kategorie vidí vybrané zboží z dané kategorie a případně také podkategorie, pokud nějaké jsou.

Uživateli je také umožněno filtrovat zboží na základě jeho parametrů. Parametry podle kterých se filtruje se dají rozdělit na druhy:
\begin{itemize}
\item \textbf{Společné pro veškeré zboží.} Sem spadá například cena výrobku, nebo značka výrobce
\item \textbf{Rozlišné pro každou kategorii.} Zde se nacházejí parametry které dávají smysl pouze v dané kategorii. Např. velikost ohniska u objektivů nemá smysl nikde jinde. Stejně tak velikost uhlopříčky nemá smysl v kategoriích kde není žádné zobrazovací zařízení.
\end{itemize}

\subsubsection{Srovnávání cen}
Každá aplikace podporuje srovnání cen u zobží. U většiny zboží v databázi je přiřazen větší počet obchodů. Díky shromáždění cen z jednotlivých obchodů je poté srovnávač schopný zobrazit u zboží nejnižší a nejvyšší cenu. U zboží je zároveň zobrazen přehled všech obhodů ve kterých je zboží nabízeno. Několik obchodů je zpravidla doporučeno srovnávačem. K tomu mohou mít obchody ještě seznam ocenění získaných srovnávačem.

Zpravidla je u obchodu zobrazena táké dostupnost zboží, případně cena dopravy. Díky těmto informacím uživatel okamžitě vidí kdy může zboží mít a kolik zaplatí navíc oproti ceně uvedené v obchodě.

Díky pravidelnému sledování cen zboží v obchodech má srovnávač chronologická data o ceně zboží. Díky této vlastnosti nabízí funcionalitu graf historie cen. V grafu je zanesena cena zboží řádově za několik uplynulých měsícu. Tento graf pomáhá uživateli v rozhodnutí o koupi. Uživatel vidí, zdali se cena právě nachází pod nebo nad dlouhodobým průměrem.

\subsubsection{Uživatelská zpětná vazba}
Srovnávače umožňují uživatelům hodnotit obchody i zobží. U obchodů se hodnotí parametry jako dodací lhůta, přehledonost obchodu a kvalita komunikace. U zboží je výběr parametrů pro hodnocení jako při filtrování zboží podle atributů. V každé kategorii se hodnotí relevantní parametry zboží. 

Hodnocení většinou probíhá na principu jedné až pěti hvězdiček, případně shrnutí kladů a záporů.

Hodnocení produktu na srovnávači Heureka.cz je vidět na obrázku \ref{fig:heureka-hodnoceni-produktu}

\begin{figure}[htb]
\begin{center}
\includegraphics[width=100mm]{./pictures/heureka-hodnoceni-produktu.png}
\caption{Hodnocení produktu Nikon 1 J1 na srovávači Heureka.cz}
\label{fig:heureka-hodnoceni-produktu}
\end{center}
\end{figure}

\subsection{Heureka}
Interaktivní nákupní rádce Heureka.cz byl založen společností Miton v roce 2007, kdy ihned zaujmul zajímavým nápadem zkombinování mnoha užitečných možností do jednoho celku. Následně rozšiřuje svou působnost i na Slovensko a zakládá slovenskou verzi Heureka.sk (založena v r. 2008). \cite{website:wiki:heureka}

\subsubsection{Funkčnost nad rámec základní funkčnosti}
Další důležitou fukčností je hlídač ceny. Přihlášenému uživateli je umožněno na zboží nastavit hlídače ceny a poté co cena v nejlevnějším obchodě klesne pod zadanou částku, uživatel dostane email s upozorněním.

\begin{itemize}
\item \textbf{Záruka vrácení peněz u některých obchodů.} U smluvních partnerů 
\item Varování před obchody, které nedodržují své obchodní podmínky
\item Sledovaní slev a novinek na trhu
\end{itemize}
\subsection{Zboží.cz}
Zboží.cz je satelitní web společnosti Seznam.cz
\footnote{Satelitní web je rozšířená a propracovanější forma tzv. microsite (jinak také minisite či weblet), jde o speciální samostatný web plnící funkce, které se na hlavní webovou prezentaci nehodí, nebo s ní dokonce nesouvisejí.}
. 

%\section{NextTag}
% \section{WunderList}

\subsection{PriceGrabber}
PriceGrabber je služba pro srovnávání cen. Jejími partnery je více než 13000 prodejců. Poskytuje volně informace o milionech produktů ve více než 25 kategoriích. Společnost také slouží jako datový zdroj pro další prodejní služby jako AOL Shopping, Bing Shopping aj. \cite{website:wiki:pricegrabber}

Price grabber byl první srovnávač, který zahrnul informace o daních a poplatcích za přepravu do srovnávání. \cite{website:wiki:pricegrabber}

\subsubsection{Funkčnost nad rámec základní funkčnosti}
Price graber obsahuje funkčnost v čechách poprvé zavedenou portálem Slevomat.cz\footnote{Slevomat je český server hromadného nakupování založený Tomášem Čuprem, Petrem Bartošem a Romanou Sudovou.}. Tato funkčnost se nazýva "Local Deals" a umožňuje uživateli pořizovat zvýhodněné služby a/nebo zboží.

PriceGrabber obsahuje velké množství tzv. "Buying Guiedes" tedy příručky k nákupu, které poskytují základní rady a doporučení pro koupi daného typu zboží. Obsahuje tedy například příručku pro nákup klimatizace.

\section{Aplikace s žádanou funkčností}
Tato kapitola popisuje několik aplikací, které svým zaměřením nesplňují přímo podmínky této práce, ale část jejich funkcionality by byla pro výslednou aplikaci přínosem.

\subsection{Astrid}
Astrid je aplikace pro správu úkolů. Mezi její klíčové funkce patří vytváření, úprava a kategorizace úkolů. Tato aplikace byla původně navržena pro mobilní zařízení a operační systém Android, nyní má aplikace také webové rozhranní, jehož funkcinalita je předmětem této kapitoly.

Základní obrazovku aplikace je možné vidět na obrázku \ref{fig:astrid}

\begin{figure}[htb]
\begin{center}
\includegraphics[width=130mm]{./pictures/astrid.png}
\caption{Základní strana webové aplikace Astrid.com}
\label{fig:astrid}
\end{center}
\end{figure}

Ná základní straně je vidět jednoduchost aplikace. Přidání úkolu proběhne pomocí vyplnění názvu úkolu a stisknutí tlačítka \emph{Add a Task}. Tím se úkol zobrazí v seznamu ostatních úkolů a jeho detaily už je pak možné nastavit stejně jako u všech ostatních úkolů.

Editace úkolu se provede kliknutím na daný úkol v seznamu. V pravé části obrazovky se zobrazí panel s detailními informacemi o úkolu.

Hlavní seznam úkolů je možné řadit pomocí několika kritérií. Mimo standardní kritéria jako priorita a datum přidání je možné řadit také manuálně (tzv. \emph{Manual order}). Uživatel může kliknout na položku na seznamu a přetáhnout ji mezi jakékoli 2 jiné položky, nebo na začátek/konec seznamu. Takto je uživateli umožněno seřadit si úkoly dle libosti.

Každý úkol náleží do 0-n seznamů. Tyto seznamy je možné libovolně výtvářet a mazat. Přidáváním přání do seznamů je uživateli umožněna další úroveň kategorizace. U každého seznamu se ukazuje panel s informacemi o aktuálních aktivitách v seznamu (např. vytvoření, dokončení a úprava úkolu).

Mazání přání probíhá pomocí tlačítka \emph{Delete Task} v panelu s informacemi o úkolu. Tímto tlačítkem se úkol okamžitě smaže, ale uživateli se zobrazí upozornění v horní části obrazovky. V tomto upozornění je také tlačítko na vrácení smazání přání viz. obrázek \ref{fig:astrid-undo}

\begin{figure}[htb]
\begin{center}
\includegraphics[width=130mm]{./pictures/astrid-undo.png}
\caption{Panel informující o smazání úkolu umožňuje vrátit operaci zpět}
\label{fig:astrid-undo}
\end{center}
\end{figure}

Toto je pro uživatele přijemnější způsob než klasické potvrzení smazání pomocí dialogu. Uživatel v naprosté většině případů má v úmyslu úkol smazat, ale pokud to udělá omylem může smazání bezpečně vrátit.

Aplikace umožňuje také poslat uživateli email s upozorněním na blížící se termín splnění přání.

Poslední zajímavá funkčnost je \emph{Remind Me button} tedy tlačítko, které si může jakýkoli uživatel umístit na webové stránky a pokud na něj klikne návštěvník této stránky, je mu automaticky přidán úkol do seznamu úkolů. HTML kód tlačítka se vytvoří pomocí jednoduchého formuláře, do kterého se zadá nadpis úkolu, za jak dlouho (popř. kdy) má být úklol splňěn, název internetové stránky, na které je tlačítko umístěno a URL které se úkolu týká.

Technicky je celá aplikace řešena pomocí AJAX. To znamená, že se stránka nenačítá znovu po každém kliknutí na tlačítko/odkaz, ale pouze se doplňují data do potřebných částí stránky. K určení cesty na stránce se používá tzv. identifikátor fragmentu\footnote{Krátký řetězec znaků označující zdroj, který je podřízený jinému, primárnímu, zdroji. Primární zdroj je určen URI, identifikátor fragmentu je od URI oddělen znakem \#.}. Frontend aplikace je postaven na technologii twitter bootstrap.
%https://twitter.com/astrid/status/222836247498985472

\section{Technologie pro vývoj aplikace}
Tato část rešerše se zabývá technologiemi, ve kterých je možné implementovat webovou aplikaci. 
\subsection{Java}
Java je objektově orientovaný programovací jazyk, který vyvinula firma Sun Microsystems a představila 23. května 1995.

Java je jedním z nejpoužívanějších programovacích jazyků na světě. Podle Tiobe indexu je Java nejpopulárnější programovací jazyk.[1] Díky své přenositelnosti je používán pro programy, které mají pracovat na různých systémech počínaje čipovými kartami (platforma JavaCard), přes mobilní telefony a různá zabudovaná zařízení (platforma Java ME), aplikace pro desktopové počítače (platforma Java SE) až po rozsáhlé distribuované systémy pracující na řadě spolupracujících počítačů rozprostřené po celém světě (platforma Java EE). Tyto technologie se jako celek nazývají platforma Java. Dne 8. května 2007 Sun uvolnil zdrojové kódy Javy (cca 2,5 miliónů řádků kódu) a Java bude dále vyvíjena jako open source.
\subsubsection{Frameworky pro tvorbu webových aplikací}
\paragraph{Spring MVC}
\subsection{Ruby}
Ruby je dynamický, reflexivní, objektově orientovaný programovací jazkyk, který kombinuje syntaxi inspirovanou jazykem Perl s funkcemi podobnými Smalltalku. Ruby je také ovlivněný jazyky Eiffel a Lisp. Ruby byl prvotně navržen a vyvinut v polovině devadesátých let minulého století japoncem Yukihiro "Matz" Matsumoto.

Ruby podporuje několik programovacích paradigmat. Mimo jiné funkcionální, objektově orientované, imperativní a reflektivní. Ruby má dynamické typování a automatickou správu paměti. Je tedy v mnohých aspektech podobný jazykům Smalltalk, Python, Perl, Lisp, Dylan, Pike a CLU.

Současná verze jazyka je 1.9.3.
\subsubsection{Frameworky pro tvorbu webových aplikací}
\paragraph{Ruby on Rails}
Ruby on Rails je framework pro vývoj webových aplikací napojených na databázi, používající návrhový vzor Model-view-controller. Vytvořil jej dánský programátor David Heinemeier Hansson při práci na projektu Basecamp.

Vše v Rails je založeno na jazyce Ruby. Na jazyce Ruby je založen Ajax v šablonách (view), odpovědi v controllerech i architektura aplikace v modelech obalujících databázi. Ke spuštění aplikace je třeba jen databáze.

Mezi základní princip Rails patří Konvence má přednost před konfigurací, tedy že programátor konfiguruje pouze ty části aplikace, které se liší od běžného nastavení. Vytvoří-li tedy např. model Person, aplikace bude data automaticky hledat v tabulce people. Chce-li, aby aplikace načítala data z tabulky staff, musí tak učinit výslovně.

Rails jsou postaveny na bázi návrhového vzoru Model-view-controller, který odděluje části aplikace zodpovědné za čtení a ukládání dat včetně manipulace s nimi (model), za zobrazení grafického rozhraní aplikace (view) a za část přijímající vstupy od uživatele a řídící zobrazení dat na výstupu (controller).

\section{Možnosti napojení aplikace na systémy srovnávání cen}
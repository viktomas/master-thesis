% !TEX root = ../DP_Vik_Tomas_2013.tex
\begin{introduction}
V dnešní době je běžné, že se zákazník před koupí produktu informuje na internetu na cenu v různých obchodech. Dříve musel na internetu použít běžný fulltextový vyhledávač a každý nalezený výsledek produktu musel rozkliknout a nalézt cenu umístěnou na stránce obchodu.

Protože se jedná o relativně složitý proces, vznikly tzv. systémy pro srovnávání cen. Ty odstraňují nutnost hledat cenu výrobku v různých zdrojích. Díky tomu může zákazník přijít na stránku srovnávače cen, nalézt produkt, který chce koupit a poté na jedné přehledné stránce vidí nejen informace o produktu, ale také srovnání cen ve všech srovnávači dostupných obchodech.

Přestože srovnávání cen je pro zákazníka usnadnění výběru, cílem této práce je poskytnou další kroky pro zjednodušení nákupu. Práce si klade za cíl navrhnout a vytvořit aplikaci, do které bude zákazníkovi umožněno vložit seznam zboží, které si přeje zakoupit. Takový seznam už některé srovnávače také podporují, ne však s dostatečnou funkcionalitou.

Uživatel bude mít možnost toto zboží (ve zbytku práce nazývané přání) v~seznamu kategorizovat, prioritizovat a různými způsoby upravovat. Aplikace dále nejen že bude u těchto přání zobrazovat vývoj ceny, ale navíc dokáže uživatele upozornit na výraznou změnu v tomto vývoji.

V práci budou nejprve detailně zpracovány vlastnosti aplikací s podobnou myšlenkou a dále vlastnosti srovnávačů cen. Na základě této rešerše bude navržena samotná aplikace. Při návrhu bude kladen důraz především na návrh grafického uživatelského rozhraní, které by mělo být intuitivní a usnadňovat uživateli co nejvíce jeho práci.

Dále je krátce popsána zajímavá část implementace a na konci práce je uživatelské vyhodnocení návrhu grafického rozhraní.
\end{introduction}
% !TEX root = ../DP_Vik_Tomas_2013.tex
\chapter{Dotazník pro vyhodnocení návrhu uživatelského rozhraní}
\label{chp:navrh-dotazniku}
Popis jakým způsobem se dospělo k navržení následujícího dotazníku je popsán v kapitole \ref{sec:metodika-dotaznik}. Následuje přesné znění dotazníku tak, jak ho obdrželi respondenti. Dotazník byl vytvořen a vyplňován elektronicky pomocí aplikace Google Drive.

\section{Vyhodnocení aplikace WishList}
Ahoj, díky za Tvůj čas, půjde přibližně o 15 min. Teď mi pomůžeš s vyhodnocením návrhu uživatelského rozhraní mojí diplomové práce, tedy jak náročné je ovládat aplikaci WishList. 

NEŽ BUDEŠ POKRAČOVAT V ČTENÍ DOTAZNÍKU, JDI PROSÍM NA STRÁNKU http://tomasvik.cz A 5-10 MINUT NA NÍ PRACUJ. POKUD UŽ TAM NEBUDEŠ MÍT CO DĚLAT, TAK MŮŽEŠ S PRACÍ NA STRÁNCE SKONČIT DŘÍVE. PAK SE VRAŤ K DOTAZNÍKU A VYPLŇ HO.

Ještě jednou díky za Tvůj čas! pánbůh ti to oplatí na dětech

\subsection{Základní dotazník}
Odpověz jak moc se ztotožňuješ s následujícími prohlášeními

\begin{enumerate}
\item \textbf{Myslím že bych rád používal systém častěji} \newline
		Vůbec nesouhlasím |1-2-3-4-5|\footnote{V elektronické verzi je možné zvolit čílo pomocí RadioButton.} Plně souhlasím
\item \textbf{Aplikace mi přišla zbytečně složitá} \newline
		Vůbec nesouhlasím |1-2-3-4-5| Plně souhlasím
\item \textbf{Aplikace se mi snadno používala} \newline
		Vůbec nesouhlasím |1-2-3-4-5| Plně souhlasím
\item \textbf{Myslím že k používání tohoto systému bych potřeboval technicky vzdělaného člověka.} \newline
		Vůbec nesouhlasím |1-2-3-4-5| Plně souhlasím
\item \textbf{Několik funkcí systému mi v aplikaci přišlo dobře umístěno} \newline
		Vůbec nesouhlasím |1-2-3-4-5| Plně souhlasím
\item \textbf{Myslím že v systému bylo příliš nesrovnalostí} \newline
		Vůbec nesouhlasím |1-2-3-4-5| Plně souhlasím
\item \textbf{Dokážu si představit že většina lidí by se naučila používat tuto aplikaci rychle} \newline
		Vůbec nesouhlasím |1-2-3-4-5| Plně souhlasím
\item \textbf{Přišel jsem na to že systém je příliš složitý na používání} \newline
		Vůbec nesouhlasím |1-2-3-4-5| Plně souhlasím
\item \textbf{Cítil jsem se že mi jde ovládání aplikace dobře} \newline
		Vůbec nesouhlasím |1-2-3-4-5| Plně souhlasím
\item \textbf{Potřeboval bych se naučit ještě hodně věcí, než bych mohl používat systém} \newline
		Vůbec nesouhlasím |1-2-3-4-5| Plně souhlasím
\end{enumerate}

\subsection{Pokročilý dotazník}
Jestli už sem tě zdržel moc, nemusíš tyto otázky vyplňovat :)

\begin{enumerate}
\item \textbf{Pochopil/a si na co aplikace slouží?} To znamená byl ti jasný účel aplikace? \newline
		vůbec nebo s velkými problémy |1-2-3-4-5| okamžitě bez problému
\item \textbf{Bylo náročné vyhledat předmět přání?} \newline
		nepřišel jsem na to jak |1-2-3-4-5| byla to hračka
\item \textbf{Přihlásil/a, nebo zaregistroval/a ses do aplikace?}
		\begin{itemize}
		\item ano
		\item Ne, ale věděl/a jsem že to je možné
		\item ne, nevěděl/a jsem že jde
		\end{itemize}
\item \textbf{Přidal/a si přání do svého seznamu?}
		\begin{itemize}
		\item ano, jedno
		\item ano, více
		\item ne
		\end{itemize}
\end{enumerate}

\subsection{Práce s přáními}
\begin{enumerate}
\item \textbf{Zkusil si přemístit svá přání v seznamu?} To znamená přesunout jedno nad druhé a tím mu změnit prioritu?
		\begin{itemize}
		\item ano
		\item ne
		\end{itemize}
\item \textbf{Všiml/a sis panelu se štítky?} V levé části stránky byl seznam štítků, všiml/a sis ho?
		\begin{itemize}
		\item ano
		\item ne
		\end{itemize}
\item \textbf{Použil/a si seznam štítků k filtrování tvých přání?} Klikl/a jsi na nějaký štítek a tím si zobrazil/a pouze přání označená tímto štítkem?
		\begin{itemize}
		\item ano
		\item ne
		\end{itemize}
\end{enumerate}

\subsection{Práce s přáním}
\begin{enumerate}
\item \textbf{Vybral/a sis vlastní obrázek k přání při jeho přidávání?}
		\begin{itemize}
		\item Ano
		\item Ne, ale věděl/a jsem že to je možné
		\item Ne, nevěděl/a jsem že to jde
		\end{itemize}
\item \textbf{Vybral/a sis vlastní obchod k přání při jeho přidávání?}
		\begin{itemize}
		\item Ano
		\item Ne, ale věděl/a jsem že to je možné
		\item Ne, nevěděl/a jsem že to jde
		\end{itemize}
\item \textbf{Smazal/a si nějaké přání ze svého seznamu?}
		\begin{itemize}
		\item Ano
		\item Ne, ale věděl/a jsem že to je možné
		\item Ne, nevěděl/a jsem že to jde
		\end{itemize}
\item \textbf{Zobrazil/a si detail nebo upravil/a si nějaké prání?} To znamená kliknout na přání v seznamu přání a zobrazit si tím obrazovku detil přání, která vypadá podobně jako při přidávání přání.
		\begin{itemize}
		\item Ano
		\item Ne, ale věděl/a jsem že to je možné
		\item Ne, nevěděl/a jsem že to jde
		\end{itemize}
\item \textbf{Označil/a si nějaké své přání jako splněné?}
		\begin{itemize}
		\item Ano
		\item Ne, ale věděl/a jsem že to je možné
		\item Ne, nevěděl/a jsem že to jde
		\end{itemize}
\item \textbf{Přidal/a si k nějakému přání při vytváření štítek?} To znamená vložit při vytváření přání do seznamu štítků nějaké hodnoty, čímž ho vlastně zařadit do kategorie se stejným názvem jako má štítek.
		\begin{itemize}
		\item Ano
		\item Ne, ale věděl/a jsem že to je možné
		\item Ne, nevěděl/a jsem že to jde
		\end{itemize}
\end{enumerate}

\subsection{Toto je poslední stránka dotazníku}
\textbf{Napadlo, překvapilo, naštvalo, nebo potěšilo tě něco při práci s aplikací?} Prosím napiš mi to sem.
\newline
\newline
\newline
\newline

\subsection{Stránka s potvrzením odeslání dotazníku}
Toto se zobrazilo uživateli, který odeslal dotazník:

Tvoje odpovědi byly zaznamenány a já je zapracuji do výsledků své práce. Ještě poslední díky za Tvůj čas. Pokud by tě náhodou zajímaly výsledky, ozvi se na vicek22@gmail.com.
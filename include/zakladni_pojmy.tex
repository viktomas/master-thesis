% !TEX root = ../DP_Vik_Tomas_2013.tex
\chapter{Vymezení základních pojmů a zkratek}
V této kapitole je vysvětleno několik zálkladních pojmů, které jsou dále používány v celé práci.
\section{Zboží}
Zboží je hmotný statek (přírodní nebo vyrobený), který je určen k prodeji. To znamená, že zboží za určitých podmínek změní svého majitele – vlastnictví produktu přechází z prodávajícího na kupujícího. Nejčastější podmínkou pro přechod vlastnictví je zaplacení kupní ceny.
%TODO doplnit zdroj (wiki), případně sehnat nějaký lepší
\section{Přání}
Tímto označením se v celé práci rozumí entita, která je svázaná se zbožím, konkrétním uživatelem a několika dalšími entitami. Tato entita reprezentuje uživatelův zájem o zboží.

\section{Aplikace}
Aplikační software (zkráceně aplikace) je v informatice veškeré programové vybavení počítače (tj. software), které umožňuje provádět nějakou užitečnou činnost (řešení konkrétního problému, interaktivní tvorbu uživatele – např. textový procesor apod.). Aplikace využívají pro interakci s uživatelem grafické nebo textové rozhraní, případně příkazový řádek. Mezi aplikace nepatří systémový software (jádro a další součásti operačního systému, např. služba Windows, démon).

\section{Webová aplikace}
Webová aplikace v softwarovém inženýrství je aplikace poskytovaná uživatelům z webového serveru přes počítačovou síť Internet, nebo její vnitropodnikovou obdobu (intranet). Webové aplikace jsou populární především pro všudypřítomnost webového prohlížeče jako klienta. Ten se pak nazývá tenkým klientem, neboť sám o sobě logiku aplikace nezná.
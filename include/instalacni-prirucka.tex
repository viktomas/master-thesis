% !TEX root = ../DP_Vik_Tomas_2013.tex
\chapter{Instalační příručka}
Aplikace je dostupná online na adrese \url{http://tomasvik.cz} a zároveň jsou její zdrojové kódy přiloženy na CD dodávaném k této práci. Následuje postup, jak nainstalovat aplikaci na lokálním přostředí.

\section{Instalace aplikace na lokálním prostředí}
Tento postup je určený pro linuxové systémy, konkrétně systémy odvozené z distribuce Debian. Zprovoznění aplikace na jiném linuxovém systému by neměl být žádný problém. Pro zprovoznění na windows doporučuji \url{http://railsinstaller.org/} na který jsou nejlepší odezvy co se týče instalace rails.

Přestože nejsou známy problémy, které by mohly nastat při instalaci na Windows, platí následující:

\textbf{Jediným zaručeně podporovaným operačním systémem je linux s APT balíčkovacím systémem.} Aplikace byla testována a vývíjena na systému Ubuntu 12.04 LTS.

Nejdříve je důležité mít nainstalovaný balík ruby ve verzi 1.9.2 a vyšší

\lstset{language = bash, style=custom}
\begin{lstlisting}
sudo apt-get install ruby1.9.3
\end{lstlisting}

pokud příkaz

\begin{lstlisting}
ruby --version
\end{lstlisting}

nevrátí správnou verzi, je potřeba tuto verzi nastavit příkazem

\begin{lstlisting}
sudo update-alternatives --config ruby
\end{lstlisting}

poté stačí nainstalovat bundler, pokud ještě není nainstalovaný

\begin{lstlisting}
sudo gem install bundler
\end{lstlisting}

přepnout se do adrésáře zdrojových kódů diplomové práce

\begin{lstlisting}
cd ~/wishlist
\end{lstlisting}

a nainstalovat vše potřebné příkazem

\begin{lstlisting}
sudo bundle install
\end{lstlisting}

nyní stačí inicializovat databázi práce

\begin{lstlisting}
rake db:setup
\end{lstlisting}

a spusti aplikaci pomocí příkazu 

\begin{lstlisting}
rails s
\end{lstlisting}

Nyní je možné aplikaci nalézt na adrese \url{http://localhost:3000}

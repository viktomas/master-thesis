% !TEX root = ../DP_Vik_Tomas_2013.tex
\chapter{Vyhodnocení}
Výsledná aplikace má model otestovaný pomocí unit testů. Ty ověřují základní funkčnost algoritmů a metod pro práci s daty modelu. Funkční a integrační testy v aplikaci přítomny nejsou.

Při vývoji aplikace byl kladen důraz na návrh uživatelského rozhranní a proto je zde kladen důraz přávě na vyhodnocení tohoto návrhu.

Důvod proč vyhodnotit návrh UI je zřejmý, výsledná aplikace má uživatelské rozhraní, které vývojář podle všech dostupných informací navrhnul tak aby bylo jednoduché na používání (viz. kapitola \ref{navrh-gui}). Na konci vývojového cyklu je třeba ověřit že úsudky v návrhu byly správné.  

Během celého procesu návrhu práce byl návrh konzultován s možnými uživateli a na základě těchto konzultací a připomínek při nich získaných byl návrh upraven do finální podoby popsané v príloze \ref{chp:wireframe}.

\section{Výběr uživatelů pro testování}

\section{Způsoby vyhodnocení návrhu uživatelského rozhraní}
\subsection{Osobní konzultace}
\subsection{Sezení se audiovizuálním záznamem}
\subsection{Sledování pohybu očí}
\subsection{Dotazník}

\section{Popis metodiky zvolené pro tuto práci}

\section{Vyhodnocení dotazníku}

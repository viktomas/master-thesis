% !TEX root = ../DP_Vik_Tomas_2013.tex
\begin{conclusion}

Moderní webové aplikace na srovnávání a nakupování popsané v rešerši jsou pro zákazníka ohromnou úsporou času a peněz při nakupování. Výsledná aplikace se snažila stavět na přidané hodnotě těchto webových aplikací a jít ještě dále v usnadnění uživatelovi činnosti.

V práci byla nalezena funkcionalita, která je běžně poskytována současnými aplikacemi a dále bylo navrženo několik funkcí, díky kterým bude výsledný seznam přání pro uživatele užitečný.

Při návrhu uživatelského rozhranní bylo postupováno v souladu s Nielsenovou heuristikou\cite{molich1990improving}. A všeobecně byl kladen velký důraz na jednoduchost UI.

Implementace proběhla v nejmodernějších technologiích, což se příznivě podepsalo na vzhledu uživatelského rozhraní, jako například dynamické donačítání dat do stránek, mnoho javascriptových modálních oken atp. Vybraná metoda získávání dat (Web Scraping) sebou přinesla jak výhody, tak několik nevýhod a to především menší interaktivitu se zdrojem dat, kvůli snaze o malé zatěžování serveru aplikace pro srovnávání cen.

Závěrečnou zkouškou aplikace bylo uživatelské testování provedené formou dotazníku pro snadnou kvantifikaci výsledků. Z tohoto testování vyplynuly následující závěry. Aplikace má dobré až výborné uživatelské rozhranní, což bylo ověřeno pomocí SUS dotazníku. Návrh UI přesto nebyl zdaleka bezchybný, což se ukázalo v části dotazníku zaměřeném na konkrétní funkce aplikace. Pro několik funkcí bylo uživatelské rozhranní navržené nedostatečně výrazně a uživatelé tyto funkce přehlédli. Dokonce ani navržení dotazníku se neobešlo bez chyby a dvě otázky se ukázaly nešťastně zforulované.

Celkově práce splnila své zadání a poskytla jedinečný pohled na systémy srovnávání cen a funkcionalitu, kterou je možné tyto velké databáze rozšířit a obohatit tak uživatelovu zkušenost.


\end{conclusion}
% !TEX root = ../DP_Vik_Tomas_2013.tex
\chapter{Návrh}
Návrh aplikace bude postupovat od shrnutí požadované funkčností přes detailní návrh grafického uživatelského rozhranní po samotný návrh implementace aplikace. V několika následujících sekcích bude tedy přesně vymezen rozsah aplikace a nastíněn způsob jejího řešení.

\section{Návrh funkcionality}
Aplikace musí splňovat požadavky dle zadání. Musí umožňovat přidávat, sledovat, prioritizovat a kategorizovat zboží, zobrazovat jeho cenu a její vývoj. Dále bude tato základní funkcionalita detailně popsána a rosšířena o další významné prvky.

V každé sekci bude funkčnost nejprve široce popsána a poté bude tučným písmem napsáno přesné znění vysledné funkcionality.

\subsection{Přihlášení/Odhlášení uživatele}
Přání je z definice vázané na uživatele a je tedy nezbytné uživatele rozlišovat. Standardní způsob jak toto rozlišení provádět ve webových aplikacích je pomocí autentizace a autorizace. Uživatel se nejprve přihlásí pomocí formuláře aplikace. Tím se ověří jeho identita. Poté se mu zobrazují informace určené pouze pro něj. Pokud před přihlášením vytvořil nějaká přání, bude dotázán, jestli chce tato přání uložit do svého seznamu. Uživatel se také může při odchodu ze systému odhlásit.

\textbf{Uživateli bude umožněno se přihlásit do aplikace. Buď pomocí uživatelského jména a hesla, nebo pomocí technologie OAuth. Pokud jako anonymní uživatel vytvořil nějaká přání, bude mu umožněno si je přidat na svůj seznam.}

\textbf{Uživateli bude umožněno se odhlásit z aplikace.}

\subsection{Registrace uživatele}
Pokud se uživatel nebude chtít od aplikace přihlásit pomocí svého OpenID v nějaké z potporovaných autorizačních autorit, bude se muset nejprve do aplikace zaregistrovat. Uživatel vyplní své klíčové údaje:

\begin{itemize}
\item \textbf{E-mail} - jednoznačný identifikátor uživatele v rámci systému
\item \textbf{Uživatelské jméno}
\item \textbf{Heslo}
\end{itemize}

Po potvrzení formuláře bude uživateli vytvořen účet v aplikaci.

\textbf{Uživatel se může zaregistrovat do aplikace pomocí poskytnutí základních údajů.} 

\subsection{Vyhledávání zboží}
\label{sec:vyhledavani}
Toto je klíčová funkcionalita aplikace. V rešerši v předchozí kapitole bylo zjištěno, že vyhledávání zboží je jedna ze základních funkčností aplikací pro srovnávání zboží.Výsledná aplikace tedy bude umožňovat vyhledávání zboží.

\textbf{Uživateli bude umožněno zadat do aplikace vyhledávaný termín jako řetězec a aplikace najde jako výsledek zboží, které v názvu obsahuje tento řetězec. Pokud žádné takové zboží nebude nalezeno, uživateli bude tato informace sdělana a připadně mu bude nabídnuto nějaké zboží, které má hledaný termín v popisu.}

\subsection{Přidání a editace přání}
\label{sec:pridani-prani}
Uživateli bude umožňeno přidávat zboží do seznamů v podobě takzvaných \emph{přání}. Samotná databáze zboží zůstane odděleně a běžný uživatel\footnote{Jedná se o uživatele, který je klientem aplikace, další druhy uživatelů mohou být administrátor, nebo internetový obchod} do samotného zboží nebude zasahovat.

Uživateli bude umožněno do seznamu přidat maximálně 20 přání. Omezený počet přání bude usnadňovat uživateli koncentraci a usnadní mu rozhodování\cite{iyengar2004much}. Počet byl daný odhadem a při využívání aplikace bude upraven podle preferencí uživatelů.

Přání bude obsahovat následující základní informace:

\begin{itemize}
\item \textbf{Vazba na zboží} - přání se z definice váže na nějaké zboží. Z tohoto zboží si přání take převezme název.
\item \textbf{Vazba na uživatele} - přání se z definice váže ke konkrétnímu uživateli.
\item \textbf{Obrázek} - obrázek přiřazený k přání pro jeho snazší vizualizaci směrem k uživateli
\item \textbf{Obchod} - konkrétní obchod, ve kterém si uživatel vybral zboží nakoupit. Používá se pro sledování ceny přání.
\item \textbf{Štítky} - libovolný počet štítků, které uživateli umožňují \textbf{kategorizovat} přání.
\item \textbf{Priorita} - informace o míře priority, tedy jak je přání důležité pro uživatele.
\end{itemize}

\textbf{Uživateli bude umožňěno vytvářet a upravovat přání. Vytvoření bude probíhat pomocí zvolení zboží ve výsledku vyhledávání (viz. kapitola \ref{sec:vyhledavani}). Následně uživatel vyplní informace o přání do vhodného formuláře. Podobný formulář bude sloužit také k editaci přání. Uživateli bude umožněno přidat maximálně 20 přání, po překročení tohoto limitu bude dotázán, jestli chce nově přidaným přáním nahradit nějaké současné přání.}

\subsubsection{Výběr obrázku}
Při vytváření nebo editaci přání je nutné umožnit uživateli vybrat si ze všech obrázků, které jsou v databázi dostupné k danému zboží. Jako první bude uživateli nabídnut nejoblíbenější obrázek.

\subsubsection{Výběr obchodu}
Při vytváření nebo editaci přání je nutné umožnit uživateli vybrat si ze všech obchodů, které jsou v databázi dostupné k danému zboží. Při výběru těchto obchodů musí být uživateli zobrazena aktualní cena zboží v daném obchodu.

\subsection{Mazání přání}
Pokud se uživatel rozhodne, že přání už déle není aktualní, může ho smazat. Smazáním ho odstraní nenávratně ze seznamu svých přání.

\textbf{Uživateli bude umožněno odstranit přání ze svého seznamu.}

\subsection{Splnění přání}
Uživatel může označit přání za splněné. Timto se přání odstraní ze standardního seznamu a přesune se do seznamu splněných přání, u kterých už se nesleduje cena. Tato přání zůstávají v sýstému pouze jako informace pro uživatele.

\textbf{Uživateli je umožněno označit přání jako splněné. Tímto označením se přání přesune do seznamu splněných přání.}

\subsection{Změna priority přání}
Jednotlivým přáním je při vytvoření přiřazena uživatelem nadefinovaná, orientační, priorita. Tuto prioritu může uživatel měnit. Přání jsou v seznamech řazena podle priority a tedy po změně priority se změní řazení přání.

\textbf{Uživatel může měnit prioritu přání.}

\subsection{Zobrazení štítků}
\label{sec:zobrazeni-stitku}
Jak bylo napsáno v sekci \ref{sec:pridani-prani}, přání je možné kategorizovat pomocí tzv. štítků. Uživatel musí mít možnost prohlédnout si všechny štítky, které přidal k přáním na jednom místě. Zároveň by u štítků měla být zobrazena informace o tom, kolik přání je daným štítkem označeno. Tato informace může sloužit jako ukazatel důležitosti/četnosti štítku.

\textbf{Uživatel si může zobrazit přehled všech štítků, které přiřadil k jednotlivým přáním.}

\subsection{Zobrazení přehledu přání}
Všechna přání, která uživatel vytvoří musí být nasledně uživateli nějakým způsobem zobrazována. Přání mohou být zobrazena buď všechna najednou, nebo mohou být zobrazena pouze přání s konkrétním štítkem. Tento omezený seznam přání se zobrazí pomocí zvolení nějakého štítku ze seznamu štítků (viz. kapitola \ref{sec:zobrazeni-stitku}).

\textbf{Uživatel si může zobrazit přehled buď všech přání, nebo přání obsahujíc konkrétní štítek.}

\subsection{Sledování průběhu cen}
Systém bude schopný u veškerého zboží sledovat cenu v jednotlivých obchodech. Vývoj ceny bude vhodným způsobem zobrazovat uživateli u jeho přání. Zároveň bude systém schopný upozornit uživatele na prudké změny ceny jeho přání (primárně snížení cen).

\subsection{Nezahrnuté funkce z rešerše}
Některé funkce, které byly zmíněny v rešerši nebudou v aplikaci z různých důvodů zahrnuty. Tyto funkce zde budou vyčteny a u každé bude uveden důvod jejího vyloučení z funkčních požadavků.

\subsubsection{Mazání bez potvrzování}
V kapitole \ref{sec:astrid} bylo popsáno mazání úkolů ze seznamu. Toto mazání fungovalo tak, že uživatel kliknul na tlačítko smazat a úkol se okamžitě smazal. Uživateli se pouze v horní části stránky zobrazil panel, kde mohl akci vrátit zpět.

Tento způsob mazání je vhodný pro manipulaci s velkým množstvím entit. Ve výsledné aplikaci maximální počet 20 přání a od uživatele se neočekává, že by je vytvářel a mazal tak často, aby mazání bez potvrzování přineslo výraznou přidanou hodnotu. Naopak chování aplikace by se odlišovalo od standardu a to by ve výsledku mohlo uškodit.

\subsubsection{Řazení podle více kritérií}
V rešerši bylo zmíněno, že mnoho aplikací umožňuje řadit své entity podle vícero kritérií. Logickými kritériemi u přání by bylo datum přidání, cena a např. míra změny ceny.

Tato funcionalita není ve výsledné aplikaci nezbytná, opět se počítá s tím, že přání bude méně než 20. Přání s aktuální výhodnou cenou uvidí uživatel hned po přihlášení a v přehledech mu budou řazena podle priority.

\section{Návrh grafického uživatelského rozhraní}
Grafické uživatelské rozhranní (dále GUI) usnadňuje používání aplikace pomocí přezentování informací formou, která je snadná na osvojení a manipulaci s informacemi. Použití vizuálníhch prvků (přepínačů, tlačítek, posuvníků, atp.) usnadňuje uživateli učení tím, že mu poskytuje intuitivní rozhranní pro práci s aplikací. Špatný návrh GUI může aplikaci uškodit, znepřehlední dodávané informace a neodhalí uživateli všechnu funkcionalitu. Tím uživatele zpravidla donutí k memorizování kroků k průchodu běžnými scénáři\cite{toby2001expgui}.

Dobrý design GUI se vyznačuje tím, že po uživateli nevyžaduje žádné memorizování zacházení s aplikací. GUI by přesto mělo umožňovat zrychlený průchod aplikací pomocí zkratek\cite{toby2001expgui}.

\subsection{Proces návrhu}
Při návrhu uživatelského rozraní se nejprve vytvoří seznam operací, které bude možné s aplikací provádět, tzv. task list. Tyto operace se poté zahrnou do skupin podle typu (například ovládání hlavního okna, přihlašování a registrace atp.) a z takto zpracovaných operací se vytvoří takzvaný task graph, tedy diagram, na kterém bude přesně zobrazeno, kdy k dané operaci může nastat a jaký bude mít následek.

Podle tohoto grafu poté bude vytvořen tzv. wireframe, tedy nakreslené obrazovky aplikace s rozvržením informací a ovládacích prvků.

\subsection{Seznam operací}
Operace se dělí na 
\begin{multicols}{2}
\begin{itemize}
\item \namedlabel{op-1}{Návrat na domovskou stránku}
\item Zobrazení domovské stránky
\item Zobrazení vyhledávání produktu
\item Vyhledání produktu
\item Přidání přání
\item Výběr obchodu s produktem přání
\item Výběr obrázku přání
\item Uložení přidaného přání
\item Zobrazení všech přání
\item Změna priority přání
\item Zorazení editace/detailu přání
\item Označení přání jako splněné
\item Smazání přání
\item Zobrazení dialogu při mazání přání
\item Potvrzení dialogu pro mazání přání
\item Zamítnutí dialogu pro mazání přání
\item Zobrazení seznamu tagů
\item Výběr tagu
\item Přihlášení uživatele pomocí uživ. jména a hesla
\item Přihlášení uživatele pomocí Google
\item Odhlášení uživatele
\item Zobrazení registrace uživatele
\item Zobrazení dialogu na přidání přání nepřihlášeného uživatele
\item Potvrzení dialogu na přidání přání nepřihlášeného uživatele
\item Zamítnutí dialogu na přidání přání nepřihlášeného uživatele
\end{itemize}
\end{multicols}

\subsection{Skupiny operací}

\subsection{Diagram operací}

\subsection{Wireframe}
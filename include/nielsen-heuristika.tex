% !TEX root = ../DP_Vik_Tomas_2013.tex
\chapter{Jednoduchý překlad Nielsenovy heuristiky}
V této příloze je přeložen význam deseti klíčových bodů Nielsenovy heuristiky.
\begin{enumerate}
\item \textbf{viditelnost stavu systému} – systém by měl být schopen uživateli vždy poskytnout v dostatečně krátkém časovém horizontu srozumitelnou zpětnou vazbu o svém stavu, aby hodnotitel věděl, co se systémem právě provádí\cite{thesis:flamik} 

\item \textbf{shoda systému a reálného světa} – systém by měl hovořit uživatelským jazykem. Měl by používat slova a fráze uživatelsky známé a neměl by naopak využívat obraty a slova, která mají srozumitelný význam jen pro návrháře a jeho autory. Dále by měl využívat zvyklosti reálného světa a informace zobrazovat v logickém pořadí\cite{thesis:flamik} 

\item \textbf{svobodné akce uživatele} – uživatel je schopen omylem si zvolit funkci systému, kterou nepotřebuje. Proto je nutné jasně naznačit „únikové místo“ pro rychlé opuštění z nechtěného stavu, ato bez nutnosti komplikovaného dialogu se systémem. Z tohoto důvodu by měl systém poskytovat funkce typu „Zpět“ a „Znovu“\cite{thesis:flamik} 

\item \textbf{konzistence a standardy} – uživatel by neměl být nucen přemýšlet, zda různé situace, akce nebo slova znamenají totéž. Systém by se měl řídit konvencemi dané platformy\cite{thesis:flamik} 

\item \textbf{prevence chyb} - je lepší mít systém, který předchází chybám, než systém, který chyby pouze ohlašuje chybovými zprávami\cite{thesis:flamik} 

\item \textbf{preference rozpoznání před vzpomínáním} – veškeré objekty a volby akcí systému musí být viditelné a srozumitelné. Uživatel by neměl být nucen pamatovat si informace z jedné části dialogu se systémem pro práci s další částí systému. Návod k používání systému by měl být v případě potřeby snadno dostupný\cite{thesis:flamik} 

\item \textbf{flexibilita a efektivnost použití} – nástroje k urychlení práce se systémem by měly být k dispozici pro zkušené uživatele. Pro nováčky by však měly být skryté. Systém by měl být přizpůsobivý z hlediska rychlosti a snadnosti ovládání jak pro začátečníky, tak pro časté a zkušené uživatele. Uživatel by měl mít tedy rychlý přístup k těm funkcím systému, které často využívá, proto by se měla rozlišovat podpora pro různé uživatele\cite{thesis:flamik} 

\item \textbf{efektivní informační design} – dialog by neměl obsahovat informace, které nejsou podstatné pro funkci systému nebo jsou potřeba jen zřídka. Každá nadbytečná informace soupeří o pozornost uživatele s informacemi podstatnými a snižuje jejich relativní viditelnost. Krátké řádky a odstavce jsou čitelnější. Grafy, tabulky, seznamy jsou reprezentativnější než rozsáhlý popis v uceleném textu\cite{thesis:flamik} 

\item \textbf{pomoc při rozpoznávání, stanovení chyb a následné zotavení} – zpráva o chybě systému by neměla být vyjádřena v heslech či kódech, ale v přirozeném jazyce vystihující daný problém a s konstruktivní nabídkou k řešení problému\cite{thesis:flamik} 

\item \textbf{nápověda a dokumentace orientovaná na úkoly} – ideálním systémem je takový systém, který ke svému používání nepotřebuje žádnou nápovědu ani dokumentaci, ale měl by je však vždy obsahovat. Každá informace v tomto dokumentu by měla být snadno dohledatelná. Měl by se zaměřovat na pomoc při řešení úkolů, které uživatel provádí nebo chce provést a měl by také obsahovat konkrétní kroky k jejich provedení. Neměl by být však příliš obsáhlý.\cite{thesis:flamik}

\end{enumerate}